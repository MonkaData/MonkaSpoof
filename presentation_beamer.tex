% Beamer presentation for TIPE – Transpilation mini-OCaml -> C
% Author: Candidat 22054

% Compile with: latexmk -pdf presentation_beamer.tex

\documentclass[aspectratio=43]{beamer}

% -----------------------------------------------------------------------------
% PACKAGES
% -----------------------------------------------------------------------------
\usepackage[utf8]{inputenc}
\usepackage[T1]{fontenc}
\usepackage[french]{babel}
\usepackage{csquotes}
\usepackage{hyperref}
\usepackage{graphicx}
\usepackage{listings}
\usepackage{hyphenat}
\usepackage{tikz}
\usetikzlibrary{arrows.meta,positioning,shapes,trees,backgrounds}

% -----------------------------------------------------------------------------
% LISTINGS STYLE
% -----------------------------------------------------------------------------
\lstdefinestyle{ocaml}{language=[Objective]Caml,basicstyle=\ttfamily\small,
  keywordstyle=\bfseries\color{blue},commentstyle=\itshape\color{gray},tabsize=2}
\lstdefinestyle{c}{language=C,basicstyle=\ttfamily\small,
  keywordstyle=\bfseries\color{purple},commentstyle=\itshape\color{gray},tabsize=2}
\lstdefinestyle{ebnf}{basicstyle=\ttfamily\small,columns=fullflexible,keepspaces=true}

% -----------------------------------------------------------------------------
% METADATA
% -----------------------------------------------------------------------------
\title{Transpilation de mini-OCaml vers C}
\subtitle{Lexing, Parsing, Sémantique et Génération}
\author{Candidat n\textsuperscript{o}~22054}
\institute{CPGE -- TIPE Informatique}
\date{Session~2025}

\begin{document}

% Title page
\begin{frame}[plain]
  \titlepage
\end{frame}

% Plan
\begin{frame}{Plan}
  \tableofcontents
\end{frame}

% -----------------------------------------------------------------------------
\section{Analyse Lexicale}
% -----------------------------------------------------------------------------
\begin{frame}{1. Analyse Lexicale – Objectif}
Découper le flux de caractères en \textbf{tokens} : mots-clés, identifiants, littéraux, opérateurs, ponctuation.
\end{frame}

\begin{frame}{1.1 Classes de Tokens}
\begin{itemize}
  \item Mots-clés : \texttt{let, in, if, then, else, rec, for, while}
  \item Identifiants : \texttt{[a-zA-Z\_][a-zA-Z0-9\_']*}
  \item Littéraux :
    \begin{itemize}
      \item Entiers : \texttt{[0-9]+}
      \item Booléens : \texttt{true | false}
      \item Chaînes : \verb|"([^"\\]|\\.)*"|
    \end{itemize}
  \item Opérateurs/Ponctuation : \texttt{+, -, *, /, =, ->, ::, ;, (, ), \{, \}}
\end{itemize}
\end{frame}

\begin{frame}[fragile]{1.2 OCamllex – Règles \& Regex}
\begin{columns}
\column{0.5\textwidth}
Extrait \texttt{lexer.mll} :
\begin{lstlisting}[style=ocaml]
rule token = parse
  | [' '\t'\n']+                     { token lexbuf }
  | "let"                            { LET }
  | ['0'-'9']+ as i                  { INT(int_of_string i) }
  | [a-zA-Z_][a-zA-Z0-9_']* as id    { IDENT id }
  | "(*" [^*]* "*)"                  { token lexbuf }
  | _                                { failwith "lex error" }
\end{lstlisting}
\column{0.45\textwidth}
Schéma pipeline :
\begin{tikzpicture}[>=Stealth,node distance=1.2cm]
  \node[draw,fill=orange!20,rounded corners] (r) {Regex};
  \node[draw,fill=cyan!20,right=of r] (nfa) {NFA};
  \node[draw,fill=green!20,right=of nfa] (dfa) {DFA};
  \node[draw,fill=yellow!20,right=of dfa] (min) {DFA minimisé};
  \draw[->] (r)--(nfa)--(dfa)--(min);
\end{tikzpicture}
\end{columns}
\end{frame}

% -----------------------------------------------------------------------------
\section{Analyse Syntaxique}
% -----------------------------------------------------------------------------
\begin{frame}{2. Analyse Syntaxique – Objectif}
Transformer la liste de tokens en \texttt{AST} représentant la structure du programme.
\end{frame}

\begin{frame}[fragile]{2.1 Grammaire EBNF}
Définition EBNF avec priorités :
\begin{lstlisting}[style=ebnf]
expr ::= "let" ident "=" expr "in" expr
       | "if" expr "then" expr "else" expr
       | expr "*" expr
       | expr "+" expr
       | ident
       | int
       | "(" expr ")"
\end{lstlisting}
\end{frame}

\begin{frame}[fragile]{2.2 Menhir – parser.mly}
\begin{columns}
\column{0.48\textwidth}
Règles \& actions :
\begin{lstlisting}[style=ocaml]
%token LET IN IF THEN ELSE
%token PLUS TIMES EQ
%token <int> INT
%start <expr> main
%%
expr:
  | LET IDENT EQ expr IN expr  { ELet($2,$4,$6) }
  | IF expr THEN expr ELSE expr { EIf($2,$4,$6) }
  | expr PLUS expr             { EBinop(Add,$1,$3) }
\end{lstlisting}
\column{0.48\textwidth}
AST correspondant (\texttt{ast.ml}) :
\begin{lstlisting}[style=ocaml]
type expr =
  | EInt   of int
  | EVar   of string
  | EBinop of binop*expr*expr
  | EIf    of expr*expr*expr
  | ELet   of string*expr*expr
\end{lstlisting}
\end{columns}
\end{frame}

% -----------------------------------------------------------------------------
\section{Analyse Sémantique}
% -----------------------------------------------------------------------------
\begin{frame}{3. Analyse Sémantique – Objectif}
Vérifier portées, types, et annoter l'AST.
\end{frame}

\begin{frame}{3.1 Table des symboles}
Gestion de la portée via une pile d'environnements :
\begin{tikzpicture}[>=Stealth,node distance=1cm]
  \node[draw,rounded corners,fill=purple!20] (e0){Env global};
  \node[draw,rounded corners,fill=purple!30,below=of e0] (e1){Env (let ...)};
  \node[draw,rounded corners,fill=purple!40,below=of e1] (e2){Env (nested)};
  \draw[->] (e0)--(e1)--(e2);
\end{tikzpicture}
\end{frame}

\begin{frame}[fragile]{3.2 Typage \& Unification}
Extraction d'équations puis unification :
\begin{lstlisting}[style=ocaml]
| EBinop(Add,e1,e2) ->
    let t1 = infer env e1 in
    let t2 = infer env e2 in
    if t1 = TyInt && t2 = TyInt then TyInt
    else failwith "Type error"
\end{lstlisting}

Schéma unification :
\begin{tikzpicture}[>=Stealth,node distance=1.5cm]
  \node[draw,ellipse,fill=green!20] (eq){t1=int};
  \node[draw,ellipse,fill=green!20,right=of eq] (eq2){t2=int};
  \draw[<->] (eq)--(eq2) node[midway,above]{unif};
\end{tikzpicture}
\end{frame}

% -----------------------------------------------------------------------------
\section{Génération de Code}
% -----------------------------------------------------------------------------
\begin{frame}{4. Génération de Code – Objectif}
Traduire l'AST ou IR annoté en code C lisible et efficace.
\end{frame}

\begin{frame}{4.1 IR Intermédiaire}
Instructions simples and labels :
\begin{tikzpicture}[>=Stealth,node distance=1.6cm]
  \node[draw,rounded corners,fill=yellow!20] (ast){AST annoté};
  \node[draw,rounded corners,fill=orange!20,right=of ast] (ir){IR};
  \node[draw,rounded corners,fill=blue!20,right=of ir] (c){Code C};
  \draw[->] (ast)--(ir)--(c);
\end{tikzpicture}
\end{frame}

\begin{frame}[fragile]{4.2 Extrait C généré}
\begin{lstlisting}[style=c]
#include <stdio.h>
int fib(int n){
  if(n<=1) return n;
  return fib(n-1)+fib(n-2);
}
int main(){return fib(10);} 
\end{lstlisting}
\end{frame}

% -----------------------------------------------------------------------------
\section{Performances}
% -----------------------------------------------------------------------------
\begin{frame}{Performances – Fibonacci}
\begin{itemize}
  \item OCaml : 0,76 s @ n=28
  \item C : 0,17 s @ n=28
  \item Facteur ~4,47
\end{itemize}
% \includegraphics[width=0.8\textwidth]{fib_perf.png}
\end{frame}

\begin{frame}{Graphique Temps – n=1..1000}
Formules :
$$t_C(n)=0.17\frac{n}{28},\quad t_{OCaml}(n)=0.76\frac{n}{28}$$
\end{frame}

% -----------------------------------------------------------------------------
\section{Conclusion}
% -----------------------------------------------------------------------------
\begin{frame}{Conclusion}
\begin{itemize}
  \item Pipeline complet : Lexing → Parsing → Sémantique → Génération
  \item Techniques classiques : Automates, LALR, Unification, IR
  \item Perspectives : optimisation, support GC, constructions avancées
\end{itemize}
\end{frame}

\begin{frame}{Questions ?}
  \centering Merci de votre attention !
\end{frame}

\end{document}
